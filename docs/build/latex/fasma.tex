%% Generated by Sphinx.
\def\sphinxdocclass{report}
\documentclass[letterpaper,10pt,english]{sphinxmanual}
\ifdefined\pdfpxdimen
   \let\sphinxpxdimen\pdfpxdimen\else\newdimen\sphinxpxdimen
\fi \sphinxpxdimen=.75bp\relax

\PassOptionsToPackage{warn}{textcomp}
\usepackage[utf8]{inputenc}
\ifdefined\DeclareUnicodeCharacter
% support both utf8 and utf8x syntaxes
  \ifdefined\DeclareUnicodeCharacterAsOptional
    \def\sphinxDUC#1{\DeclareUnicodeCharacter{"#1}}
  \else
    \let\sphinxDUC\DeclareUnicodeCharacter
  \fi
  \sphinxDUC{00A0}{\nobreakspace}
  \sphinxDUC{2500}{\sphinxunichar{2500}}
  \sphinxDUC{2502}{\sphinxunichar{2502}}
  \sphinxDUC{2514}{\sphinxunichar{2514}}
  \sphinxDUC{251C}{\sphinxunichar{251C}}
  \sphinxDUC{2572}{\textbackslash}
\fi
\usepackage{cmap}
\usepackage[T1]{fontenc}
\usepackage{amsmath,amssymb,amstext}
\usepackage{babel}



\usepackage{times}
\expandafter\ifx\csname T@LGR\endcsname\relax
\else
% LGR was declared as font encoding
  \substitutefont{LGR}{\rmdefault}{cmr}
  \substitutefont{LGR}{\sfdefault}{cmss}
  \substitutefont{LGR}{\ttdefault}{cmtt}
\fi
\expandafter\ifx\csname T@X2\endcsname\relax
  \expandafter\ifx\csname T@T2A\endcsname\relax
  \else
  % T2A was declared as font encoding
    \substitutefont{T2A}{\rmdefault}{cmr}
    \substitutefont{T2A}{\sfdefault}{cmss}
    \substitutefont{T2A}{\ttdefault}{cmtt}
  \fi
\else
% X2 was declared as font encoding
  \substitutefont{X2}{\rmdefault}{cmr}
  \substitutefont{X2}{\sfdefault}{cmss}
  \substitutefont{X2}{\ttdefault}{cmtt}
\fi


\usepackage[Bjarne]{fncychap}
\usepackage{sphinx}

\fvset{fontsize=\small}
\usepackage{geometry}


% Include hyperref last.
\usepackage{hyperref}
% Fix anchor placement for figures with captions.
\usepackage{hypcap}% it must be loaded after hyperref.
% Set up styles of URL: it should be placed after hyperref.
\urlstyle{same}
\addto\captionsenglish{\renewcommand{\contentsname}{Contents:}}

\usepackage{sphinxmessages}
\setcounter{tocdepth}{1}



\title{FASMA}
\date{Mar 03, 2020}
\release{2.0}
\author{FASMA}
\newcommand{\sphinxlogo}{\vbox{}}
\renewcommand{\releasename}{Release}
\makeindex
\begin{document}

\ifdefined\shorthandoff
  \ifnum\catcode`\=\string=\active\shorthandoff{=}\fi
  \ifnum\catcode`\"=\active\shorthandoff{"}\fi
\fi

\pagestyle{empty}
\sphinxmaketitle
\pagestyle{plain}
\sphinxtableofcontents
\pagestyle{normal}
\phantomsection\label{\detokenize{index::doc}}



\chapter{FASMA package}
\label{\detokenize{index:fasma-package}}

\section{Subpackages}
\label{\detokenize{index:subpackages}}

\subsection{FASMA.tests package}
\label{\detokenize{FASMA.tests:fasma-tests-package}}\label{\detokenize{FASMA.tests::doc}}

\subsubsection{Submodules}
\label{\detokenize{FASMA.tests:submodules}}

\subsubsection{FASMA.tests.test\_FASMA module}
\label{\detokenize{FASMA.tests:module-FASMA.tests.test_FASMA}}\label{\detokenize{FASMA.tests:fasma-tests-test-fasma-module}}\index{FASMA.tests.test\_FASMA (module)@\spxentry{FASMA.tests.test\_FASMA}\spxextra{module}}\index{TestAbundances (class in FASMA.tests.test\_FASMA)@\spxentry{TestAbundances}\spxextra{class in FASMA.tests.test\_FASMA}}

\begin{fulllineitems}
\phantomsection\label{\detokenize{FASMA.tests:FASMA.tests.test_FASMA.TestAbundances}}\pysiglinewithargsret{\sphinxbfcode{\sphinxupquote{class }}\sphinxcode{\sphinxupquote{FASMA.tests.test\_FASMA.}}\sphinxbfcode{\sphinxupquote{TestAbundances}}}{\emph{methodName=\textquotesingle{}runTest\textquotesingle{}}}{}
Bases: \sphinxcode{\sphinxupquote{unittest.case.TestCase}}
\index{test\_minimization() (FASMA.tests.test\_FASMA.TestAbundances method)@\spxentry{test\_minimization()}\spxextra{FASMA.tests.test\_FASMA.TestAbundances method}}

\begin{fulllineitems}
\phantomsection\label{\detokenize{FASMA.tests:FASMA.tests.test_FASMA.TestAbundances.test_minimization}}\pysiglinewithargsret{\sphinxbfcode{\sphinxupquote{test\_minimization}}}{}{}
\end{fulllineitems}

\index{test\_minimization\_values() (FASMA.tests.test\_FASMA.TestAbundances method)@\spxentry{test\_minimization\_values()}\spxextra{FASMA.tests.test\_FASMA.TestAbundances method}}

\begin{fulllineitems}
\phantomsection\label{\detokenize{FASMA.tests:FASMA.tests.test_FASMA.TestAbundances.test_minimization_values}}\pysiglinewithargsret{\sphinxbfcode{\sphinxupquote{test\_minimization\_values}}}{}{}
\end{fulllineitems}


\end{fulllineitems}

\index{TestSolarAbundance (class in FASMA.tests.test\_FASMA)@\spxentry{TestSolarAbundance}\spxextra{class in FASMA.tests.test\_FASMA}}

\begin{fulllineitems}
\phantomsection\label{\detokenize{FASMA.tests:FASMA.tests.test_FASMA.TestSolarAbundance}}\pysiglinewithargsret{\sphinxbfcode{\sphinxupquote{class }}\sphinxcode{\sphinxupquote{FASMA.tests.test\_FASMA.}}\sphinxbfcode{\sphinxupquote{TestSolarAbundance}}}{\emph{methodName=\textquotesingle{}runTest\textquotesingle{}}}{}
Bases: \sphinxcode{\sphinxupquote{unittest.case.TestCase}}
\index{test\_none\_return() (FASMA.tests.test\_FASMA.TestSolarAbundance method)@\spxentry{test\_none\_return()}\spxextra{FASMA.tests.test\_FASMA.TestSolarAbundance method}}

\begin{fulllineitems}
\phantomsection\label{\detokenize{FASMA.tests:FASMA.tests.test_FASMA.TestSolarAbundance.test_none_return}}\pysiglinewithargsret{\sphinxbfcode{\sphinxupquote{test\_none\_return}}}{}{}
\end{fulllineitems}

\index{test\_return\_values() (FASMA.tests.test\_FASMA.TestSolarAbundance method)@\spxentry{test\_return\_values()}\spxextra{FASMA.tests.test\_FASMA.TestSolarAbundance method}}

\begin{fulllineitems}
\phantomsection\label{\detokenize{FASMA.tests:FASMA.tests.test_FASMA.TestSolarAbundance.test_return_values}}\pysiglinewithargsret{\sphinxbfcode{\sphinxupquote{test\_return\_values}}}{}{}
\end{fulllineitems}


\end{fulllineitems}



\subsubsection{Module contents}
\label{\detokenize{FASMA.tests:module-FASMA.tests}}\label{\detokenize{FASMA.tests:module-contents}}\index{FASMA.tests (module)@\spxentry{FASMA.tests}\spxextra{module}}

\section{FASMA.fasma module}
\label{\detokenize{index:module-FASMA.fasma}}\label{\detokenize{index:fasma-fasma-module}}\index{FASMA.fasma (module)@\spxentry{FASMA.fasma}\spxextra{module}}

\section{FASMA.interpolation module}
\label{\detokenize{index:module-FASMA.interpolation}}\label{\detokenize{index:fasma-interpolation-module}}\index{FASMA.interpolation (module)@\spxentry{FASMA.interpolation}\spxextra{module}}\index{interpolator() (in module FASMA.interpolation)@\spxentry{interpolator()}\spxextra{in module FASMA.interpolation}}

\begin{fulllineitems}
\phantomsection\label{\detokenize{index:FASMA.interpolation.interpolator}}\pysiglinewithargsret{\sphinxcode{\sphinxupquote{FASMA.interpolation.}}\sphinxbfcode{\sphinxupquote{interpolator}}}{\emph{params}, \emph{abund=0.0}, \emph{elem=False}, \emph{save=True}, \emph{atmtype=\textquotesingle{}apogee\_kurucz\textquotesingle{}}, \emph{result=None}}{}
Function to connect all. For a given set of params, return a model atmosphere.
\begin{description}
\item[{params}] \leavevmode{[}list of length 3{]}
Teff, logg, {[}Fe/H{]}.

\item[{abund}] \leavevmode{[}float{]}
abundance of a given element to added in the atmosphere.

\item[{element}] \leavevmode{[}float{]}
any element to change abundance in the atmosphere.

\item[{save}] \leavevmode{[}bool{]}
Whether the new atmosphere should be saved. Default is True.

\item[{atmtype}] \leavevmode{[}str{]}
The atmosphere models being used. Default is apogee\_kurucz.

\item[{result}] \leavevmode{[}bool{]}
return the new atmosphere. Default is False.

\end{description}
\begin{description}
\item[{newatm}] \leavevmode{[}ndarray{]}
New interpolated atmosphere.

\end{description}

\end{fulllineitems}

\index{interpolator\_kurucz() (in module FASMA.interpolation)@\spxentry{interpolator\_kurucz()}\spxextra{in module FASMA.interpolation}}

\begin{fulllineitems}
\phantomsection\label{\detokenize{index:FASMA.interpolation.interpolator_kurucz}}\pysiglinewithargsret{\sphinxcode{\sphinxupquote{FASMA.interpolation.}}\sphinxbfcode{\sphinxupquote{interpolator\_kurucz}}}{\emph{params}, \emph{atmtype=\textquotesingle{}apogee\_kurucz\textquotesingle{}}}{}
Interpolation for Kurucz model atmospheres.
Input
—\textendash{}
params : ndarray
\begin{quote}

Stellar parameters for the Interpolation.
\end{quote}
\begin{description}
\item[{newatm}] \leavevmode{[}ndarray{]}
The interpolated atmosphere, the columns and tauross in a tuple

\end{description}

\end{fulllineitems}

\index{interpolator\_marcs() (in module FASMA.interpolation)@\spxentry{interpolator\_marcs()}\spxextra{in module FASMA.interpolation}}

\begin{fulllineitems}
\phantomsection\label{\detokenize{index:FASMA.interpolation.interpolator_marcs}}\pysiglinewithargsret{\sphinxcode{\sphinxupquote{FASMA.interpolation.}}\sphinxbfcode{\sphinxupquote{interpolator\_marcs}}}{\emph{params}, \emph{fesun=7.47}, \emph{microlim=3.0}}{}
Interpolation for marcs models. The function is taken from STEPAR
(Tabernero et al. 2019) to deal the gaps in the grid.
Input
—\textendash{}
params : ndarray
\begin{quote}

Stellar parameters for the Interpolation.
\end{quote}
\begin{description}
\item[{newatm}] \leavevmode{[}ndarray{]}
The interpolated atmosphere, the columns and tauross in a tuple

\end{description}

\end{fulllineitems}

\index{read\_model() (in module FASMA.interpolation)@\spxentry{read\_model()}\spxextra{in module FASMA.interpolation}}

\begin{fulllineitems}
\phantomsection\label{\detokenize{index:FASMA.interpolation.read_model}}\pysiglinewithargsret{\sphinxcode{\sphinxupquote{FASMA.interpolation.}}\sphinxbfcode{\sphinxupquote{read\_model}}}{\emph{fname}}{}
Read the KURUCZ model atmosphere.
\begin{description}
\item[{fname}] \leavevmode{[}str{]}
The gz file of the model atmosphere.

\end{description}
\begin{description}
\item[{model}] \leavevmode{[}ndarray{]}
The correct atmosphere, the columns and tauross in a tuple

\end{description}

\end{fulllineitems}

\index{save\_model() (in module FASMA.interpolation)@\spxentry{save\_model()}\spxextra{in module FASMA.interpolation}}

\begin{fulllineitems}
\phantomsection\label{\detokenize{index:FASMA.interpolation.save_model}}\pysiglinewithargsret{\sphinxcode{\sphinxupquote{FASMA.interpolation.}}\sphinxbfcode{\sphinxupquote{save\_model}}}{\emph{model}, \emph{params}, \emph{abund=0.0}, \emph{elem=False}, \emph{type=\textquotesingle{}apogee\_kurucz\textquotesingle{}}, \emph{fout=\textquotesingle{}out.atm\textquotesingle{}}}{}
Save the model atmosphere in the right format.
\begin{description}
\item[{model}] \leavevmode{[}ndarray{]}
The interpolated model atmosphere.

\item[{params}] \leavevmode{[}list{]}
Teff, logg, {[}M/H{]}, vt of the interpolated atmosphere.

\item[{abund}] \leavevmode{[}float{]}
abundance of a given element to added in the atmosphere.

\item[{element}] \leavevmode{[}float{]}
any element to change abundance in the atmosphere.

\item[{type}] \leavevmode{[}str{]}
Type of atmospheric parameters. Default is apogee\_kurucz

\item[{fout}] \leavevmode{[}str{]}
Name of the saved atmosphere. Default is out.atm

\end{description}

Saved atmospheric model in file.

\end{fulllineitems}

\index{solar\_abundance() (in module FASMA.interpolation)@\spxentry{solar\_abundance()}\spxextra{in module FASMA.interpolation}}

\begin{fulllineitems}
\phantomsection\label{\detokenize{index:FASMA.interpolation.solar_abundance}}\pysiglinewithargsret{\sphinxcode{\sphinxupquote{FASMA.interpolation.}}\sphinxbfcode{\sphinxupquote{solar\_abundance}}}{\emph{element}}{}
For a given atomic number, return solar abundance from Asplund et al. 2009.
\begin{description}
\item[{element}] \leavevmode{[}str{]}
The name of an element

\end{description}
\begin{description}
\item[{index}] \leavevmode{[}int{]}
The atomic number of the element

\item[{abundance}] \leavevmode{[}float{]}
The solar abundance of the atom in dex

\end{description}

\end{fulllineitems}



\section{FASMA.minimization module}
\label{\detokenize{index:module-FASMA.minimization}}\label{\detokenize{index:fasma-minimization-module}}\index{FASMA.minimization (module)@\spxentry{FASMA.minimization}\spxextra{module}}\index{MinimizeSynth (class in FASMA.minimization)@\spxentry{MinimizeSynth}\spxextra{class in FASMA.minimization}}

\begin{fulllineitems}
\phantomsection\label{\detokenize{index:FASMA.minimization.MinimizeSynth}}\pysiglinewithargsret{\sphinxbfcode{\sphinxupquote{class }}\sphinxcode{\sphinxupquote{FASMA.minimization.}}\sphinxbfcode{\sphinxupquote{MinimizeSynth}}}{\emph{p0}, \emph{xobs}, \emph{yobs}, \emph{ranges}, \emph{**kwargs}}{}
Bases: \sphinxcode{\sphinxupquote{object}}

Minimize the chi square function between a synthetic spectrum to an observed.
\index{bounds() (FASMA.minimization.MinimizeSynth method)@\spxentry{bounds()}\spxextra{FASMA.minimization.MinimizeSynth method}}

\begin{fulllineitems}
\phantomsection\label{\detokenize{index:FASMA.minimization.MinimizeSynth.bounds}}\pysiglinewithargsret{\sphinxbfcode{\sphinxupquote{bounds}}}{\emph{i}, \emph{p}}{}
Function to check if parameters during minimization are within bounds.
Input
—\textendash{}
p: parameter we want to check;
i: the index of the bounds we want to check.

\end{fulllineitems}

\index{convergence\_info() (FASMA.minimization.MinimizeSynth method)@\spxentry{convergence\_info()}\spxextra{FASMA.minimization.MinimizeSynth method}}

\begin{fulllineitems}
\phantomsection\label{\detokenize{index:FASMA.minimization.MinimizeSynth.convergence_info}}\pysiglinewithargsret{\sphinxbfcode{\sphinxupquote{convergence\_info}}}{\emph{res}}{}
Information on convergence from mpfit function. All values greater
than zero can represent success (however status == 5 may indicate failure
to converge).
If the fit is unweighted (i.e. no errors were given, or the weights
were uniformly set to unity), then .perror will probably not represent
the true parameter uncertainties.
\sphinxstyleemphasis{If} you can assume that the true reduced chi\sphinxhyphen{}squared value is unity \textendash{}
meaning that the fit is implicitly assumed to be of good quality \textendash{}
then the estimated parameter uncertainties can be computed by scaling
.perror by the measured chi\sphinxhyphen{}squared value.

res : ndarray
the result of the mpfit function

parameters : ndarray
stellar parameters or abundances with their errors

\end{fulllineitems}

\index{exclude\_bad\_points() (FASMA.minimization.MinimizeSynth method)@\spxentry{exclude\_bad\_points()}\spxextra{FASMA.minimization.MinimizeSynth method}}

\begin{fulllineitems}
\phantomsection\label{\detokenize{index:FASMA.minimization.MinimizeSynth.exclude_bad_points}}\pysiglinewithargsret{\sphinxbfcode{\sphinxupquote{exclude\_bad\_points}}}{}{}
Function to exclude points from the spectrum, e.g. lines which are not
in line list by comparing observations with the best synthetic model.

\end{fulllineitems}

\index{minimize() (FASMA.minimization.MinimizeSynth method)@\spxentry{minimize()}\spxextra{FASMA.minimization.MinimizeSynth method}}

\begin{fulllineitems}
\phantomsection\label{\detokenize{index:FASMA.minimization.MinimizeSynth.minimize}}\pysiglinewithargsret{\sphinxbfcode{\sphinxupquote{minimize}}}{}{}
Function to glue the minimization together. Set the minimization options here.

\end{fulllineitems}

\index{minimizeElement() (FASMA.minimization.MinimizeSynth method)@\spxentry{minimizeElement()}\spxextra{FASMA.minimization.MinimizeSynth method}}

\begin{fulllineitems}
\phantomsection\label{\detokenize{index:FASMA.minimization.MinimizeSynth.minimizeElement}}\pysiglinewithargsret{\sphinxbfcode{\sphinxupquote{minimizeElement}}}{}{}
\end{fulllineitems}

\index{myfunct() (FASMA.minimization.MinimizeSynth method)@\spxentry{myfunct()}\spxextra{FASMA.minimization.MinimizeSynth method}}

\begin{fulllineitems}
\phantomsection\label{\detokenize{index:FASMA.minimization.MinimizeSynth.myfunct}}\pysiglinewithargsret{\sphinxbfcode{\sphinxupquote{myfunct}}}{\emph{p}, \emph{**kwargs}}{}
Function that returns the weighted deviates (to be minimized).
\begin{description}
\item[{p}] \leavevmode{[}list{]}
Parameters to be minimized

\end{description}
\begin{description}
\item[{(y\sphinxhyphen{}ymodel)/err}] \leavevmode{[}ndarray{]}
Model deviation from observation

\end{description}

\end{fulllineitems}

\index{parinfo\_limit() (FASMA.minimization.MinimizeSynth method)@\spxentry{parinfo\_limit()}\spxextra{FASMA.minimization.MinimizeSynth method}}

\begin{fulllineitems}
\phantomsection\label{\detokenize{index:FASMA.minimization.MinimizeSynth.parinfo_limit}}\pysiglinewithargsret{\sphinxbfcode{\sphinxupquote{parinfo\_limit}}}{}{}
Smart way to calculate the bounds of each of parameters depending on
the grid of model atmospheres.

\end{fulllineitems}


\end{fulllineitems}

\index{getMac() (in module FASMA.minimization)@\spxentry{getMac()}\spxextra{in module FASMA.minimization}}

\begin{fulllineitems}
\phantomsection\label{\detokenize{index:FASMA.minimization.getMac}}\pysiglinewithargsret{\sphinxcode{\sphinxupquote{FASMA.minimization.}}\sphinxbfcode{\sphinxupquote{getMac}}}{\emph{teff}, \emph{logg}}{}
Calculate macroturbulence.
For hotter dwarfs: Doyle et al. 2014
5200 \textless{} teff \textless{} 6400 K
4.0 \textless{} logg \textless{} 4.6 dex
For cooler dwarfs: Valenti et al. 2005
For subgiants and giants (logg \textless{} 3.80 dex): Hekker \& Melendez 2007

\end{fulllineitems}

\index{getMic() (in module FASMA.minimization)@\spxentry{getMic()}\spxextra{in module FASMA.minimization}}

\begin{fulllineitems}
\phantomsection\label{\detokenize{index:FASMA.minimization.getMic}}\pysiglinewithargsret{\sphinxcode{\sphinxupquote{FASMA.minimization.}}\sphinxbfcode{\sphinxupquote{getMic}}}{\emph{teff}, \emph{logg}, \emph{feh}}{}
Calculate microturbulence.

\end{fulllineitems}



\section{FASMA.mpfit module}
\label{\detokenize{index:module-FASMA.mpfit}}\label{\detokenize{index:fasma-mpfit-module}}\index{FASMA.mpfit (module)@\spxentry{FASMA.mpfit}\spxextra{module}}
Perform Levenberg\sphinxhyphen{}Marquardt least\sphinxhyphen{}squares minimization, based on MINPACK\sphinxhyphen{}1.
\begin{quote}
\begin{quote}
\begin{quote}

AUTHORS
\end{quote}

The original version of this software, called LMFIT, was written in FORTRAN
as part of the MINPACK\sphinxhyphen{}1 package by XXX.

Craig Markwardt converted the FORTRAN code to IDL.  The information for the
IDL version is:
\begin{quote}

Craig B. Markwardt, NASA/GSFC Code 662, Greenbelt, MD 20770
\sphinxhref{mailto:craigm@lheamail.gsfc.nasa.gov}{craigm@lheamail.gsfc.nasa.gov}
UPDATED VERSIONs can be found on my WEB PAGE:
\begin{quote}

\sphinxurl{http://cow.physics.wisc.edu/~craigm/idl/idl.html}
\end{quote}
\end{quote}
\begin{description}
\item[{Mark Rivers created this Python version from Craig’s IDL version.}] \leavevmode
Mark Rivers, University of Chicago
Building 434A, Argonne National Laboratory
9700 South Cass Avenue, Argonne, IL 60439
\sphinxhref{mailto:rivers@cars.uchicago.edu}{rivers@cars.uchicago.edu}
Updated versions can be found at \sphinxurl{http://cars.uchicago.edu/software}

\end{description}
\end{quote}
\begin{description}
\item[{Sergey Koposov converted the Mark’s Python version from Numeric to numpy}] \leavevmode
Sergey Koposov, University of Cambridge, Institute of Astronomy,
Madingley road, CB3 0HA, Cambridge, UK
\sphinxhref{mailto:koposov@ast.cam.ac.uk}{koposov@ast.cam.ac.uk}
Updated versions can be found at \sphinxurl{http://code.google.com/p/astrolibpy/source/browse/trunk/}
\begin{quote}

DESCRIPTION
\end{quote}

\end{description}

MPFIT uses the Levenberg\sphinxhyphen{}Marquardt technique to solve the
least\sphinxhyphen{}squares problem.  In its typical use, MPFIT will be used to
fit a user\sphinxhyphen{}supplied function (the “model”) to user\sphinxhyphen{}supplied data
points (the “data”) by adjusting a set of parameters.  MPFIT is
based upon MINPACK\sphinxhyphen{}1 (LMDIF.F) by More’ and collaborators.

For example, a researcher may think that a set of observed data
points is best modelled with a Gaussian curve.  A Gaussian curve is
parameterized by its mean, standard deviation and normalization.
MPFIT will, within certain constraints, find the set of parameters
which best fits the data.  The fit is “best” in the least\sphinxhyphen{}squares
sense; that is, the sum of the weighted squared differences between
the model and data is minimized.

The Levenberg\sphinxhyphen{}Marquardt technique is a particular strategy for
iteratively searching for the best fit.  This particular
implementation is drawn from MINPACK\sphinxhyphen{}1 (see NETLIB), and is much faster
and more accurate than the version provided in the Scientific Python package
in Scientific.Functions.LeastSquares.
This version allows upper and lower bounding constraints to be placed on each
parameter, or the parameter can be held fixed.

The user\sphinxhyphen{}supplied Python function should return an array of weighted
deviations between model and data.  In a typical scientific problem
the residuals should be weighted so that each deviate has a
gaussian sigma of 1.0.  If X represents values of the independent
variable, Y represents a measurement for each value of X, and ERR
represents the error in the measurements, then the deviates could
be calculated as follows:
\begin{quote}

DEVIATES = (Y \sphinxhyphen{} F(X)) / ERR
\end{quote}

where F is the analytical function representing the model.  You are
recommended to use the convenience functions MPFITFUN and
MPFITEXPR, which are driver functions that calculate the deviates
for you.  If ERR are the 1\sphinxhyphen{}sigma uncertainties in Y, then
\begin{quote}

TOTAL( DEVIATES\textasciicircum{}2 )
\end{quote}

will be the total chi\sphinxhyphen{}squared value.  MPFIT will minimize the
chi\sphinxhyphen{}square value.  The values of X, Y and ERR are passed through
MPFIT to the user\sphinxhyphen{}supplied function via the FUNCTKW keyword.

Simple constraints can be placed on parameter values by using the
PARINFO keyword to MPFIT.  See below for a description of this
keyword.

MPFIT does not perform more general optimization tasks.  See TNMIN
instead.  MPFIT is customized, based on MINPACK\sphinxhyphen{}1, to the
least\sphinxhyphen{}squares minimization problem.
\begin{quote}

USER FUNCTION
\end{quote}

The user must define a function which returns the appropriate
values as specified above.  The function should return the weighted
deviations between the model and the data.  It should also return a status
flag and an optional partial derivative array.  For applications which
use finite\sphinxhyphen{}difference derivatives \textendash{} the default \textendash{} the user
function should be declared in the following way:
\begin{quote}
\begin{description}
\item[{def myfunct(p, fjac=None, x=None, y=None, err=None)}] \leavevmode
\# Parameter values are passed in “p”
\# If fjac==None then partial derivatives should not be
\# computed.  It will always be None if MPFIT is called with default
\# flag.
model = F(x, p)
\# Non\sphinxhyphen{}negative status value means MPFIT should continue, negative means
\# stop the calculation.
status = 0
return({[}status, (y\sphinxhyphen{}model)/err{]}

\end{description}
\end{quote}

See below for applications with analytical derivatives.

The keyword parameters X, Y, and ERR in the example above are
suggestive but not required.  Any parameters can be passed to
MYFUNCT by using the functkw keyword to MPFIT.  Use MPFITFUN and
MPFITEXPR if you need ideas on how to do that.  The function \sphinxstyleemphasis{must}
accept a parameter list, P.

In general there are no restrictions on the number of dimensions in
X, Y or ERR.  However the deviates \sphinxstyleemphasis{must} be returned in a
one\sphinxhyphen{}dimensional Numeric array of type Float.

User functions may also indicate a fatal error condition using the
status return described above. If status is set to a number between
\sphinxhyphen{}15 and \sphinxhyphen{}1 then MPFIT will stop the calculation and return to the caller.
\begin{quote}

ANALYTIC DERIVATIVES
\end{quote}

In the search for the best\sphinxhyphen{}fit solution, MPFIT by default
calculates derivatives numerically via a finite difference
approximation.  The user\sphinxhyphen{}supplied function need not calculate the
derivatives explicitly.  However, if you desire to compute them
analytically, then the AUTODERIVATIVE=0 keyword must be passed to MPFIT.
As a practical matter, it is often sufficient and even faster to allow
MPFIT to calculate the derivatives numerically, and so
AUTODERIVATIVE=0 is not necessary.

If AUTODERIVATIVE=0 is used then the user function must check the parameter
FJAC, and if FJAC!=None then return the partial derivative array in the
return list.
\begin{quote}
\begin{description}
\item[{def myfunct(p, fjac=None, x=None, y=None, err=None)}] \leavevmode
\# Parameter values are passed in “p”
\# If FJAC!=None then partial derivatives must be comptuer.
\# FJAC contains an array of len(p), where each entry
\# is 1 if that parameter is free and 0 if it is fixed.
model = F(x, p)
Non\sphinxhyphen{}negative status value means MPFIT should continue, negative means
\# stop the calculation.
status = 0
if (dojac):
\begin{quote}

pderiv = zeros({[}len(x), len(p){]}, Float)
for j in range(len(p)):
\begin{quote}

pderiv{[}:,j{]} = FGRAD(x, p, j)
\end{quote}
\end{quote}
\begin{description}
\item[{else:}] \leavevmode
pderiv = None

\end{description}

return({[}status, (y\sphinxhyphen{}model)/err, pderiv{]}

\end{description}
\end{quote}

where FGRAD(x, p, i) is a user function which must compute the
derivative of the model with respect to parameter P{[}i{]} at X.  When
finite differencing is used for computing derivatives (ie, when
AUTODERIVATIVE=1), or when MPFIT needs only the errors but not the
derivatives the parameter FJAC=None.

Derivatives should be returned in the PDERIV array. PDERIV should be an m x
n array, where m is the number of data points and n is the number
of parameters.  dp{[}i,j{]} is the derivative at the ith point with
respect to the jth parameter.

The derivatives with respect to fixed parameters are ignored; zero
is an appropriate value to insert for those derivatives.  Upon
input to the user function, FJAC is set to a vector with the same
length as P, with a value of 1 for a parameter which is free, and a
value of zero for a parameter which is fixed (and hence no
derivative needs to be calculated).

If the data is higher than one dimensional, then the \sphinxstyleemphasis{last}
dimension should be the parameter dimension.  Example: fitting a
50x50 image, “dp” should be 50x50xNPAR.
\begin{quote}

CONSTRAINING PARAMETER VALUES WITH THE PARINFO KEYWORD
\end{quote}

The behavior of MPFIT can be modified with respect to each
parameter to be fitted.  A parameter value can be fixed; simple
boundary constraints can be imposed; limitations on the parameter
changes can be imposed; properties of the automatic derivative can
be modified; and parameters can be tied to one another.

These properties are governed by the PARINFO structure, which is
passed as a keyword parameter to MPFIT.

PARINFO should be a list of dictionaries, one list entry for each parameter.
Each parameter is associated with one element of the array, in
numerical order.  The dictionary can have the following keys
(none are required, keys are case insensitive):
\begin{quote}
\begin{description}
\item[{‘value’ \sphinxhyphen{} the starting parameter value (but see the START\_PARAMS}] \leavevmode
parameter for more information).

\item[{‘fixed’ \sphinxhyphen{} a boolean value, whether the parameter is to be held}] \leavevmode
fixed or not.  Fixed parameters are not varied by
MPFIT, but are passed on to MYFUNCT for evaluation.

\item[{‘limited’ \sphinxhyphen{} a two\sphinxhyphen{}element boolean array.  If the first/second}] \leavevmode
element is set, then the parameter is bounded on the
lower/upper side.  A parameter can be bounded on both
sides.  Both LIMITED and LIMITS must be given
together.

\item[{‘limits’ \sphinxhyphen{} a two\sphinxhyphen{}element float array.  Gives the}] \leavevmode
parameter limits on the lower and upper sides,
respectively.  Zero, one or two of these values can be
set, depending on the values of LIMITED.  Both LIMITED
and LIMITS must be given together.

\item[{‘parname’ \sphinxhyphen{} a string, giving the name of the parameter.  The}] \leavevmode
fitting code of MPFIT does not use this tag in any
way.  However, the default iterfunct will print the
parameter name if available.

\item[{‘step’ \sphinxhyphen{} the step size to be used in calculating the numerical}] \leavevmode
derivatives.  If set to zero, then the step size is
computed automatically.  Ignored when AUTODERIVATIVE=0.

\item[{‘mpside’ \sphinxhyphen{} the sidedness of the finite difference when computing}] \leavevmode\begin{quote}

numerical derivatives.  This field can take four
values:
\begin{quote}
\begin{quote}

0 \sphinxhyphen{} one\sphinxhyphen{}sided derivative computed automatically
1 \sphinxhyphen{} one\sphinxhyphen{}sided derivative (f(x+h) \sphinxhyphen{} f(x)  )/h
\end{quote}
\begin{description}
\item[{\sphinxhyphen{}1 \sphinxhyphen{} one\sphinxhyphen{}sided derivative (f(x)   \sphinxhyphen{} f(x\sphinxhyphen{}h))/h}] \leavevmode
2 \sphinxhyphen{} two\sphinxhyphen{}sided derivative (f(x+h) \sphinxhyphen{} f(x\sphinxhyphen{}h))/(2*h)

\end{description}
\end{quote}
\end{quote}

Where H is the STEP parameter described above.  The
“automatic” one\sphinxhyphen{}sided derivative method will chose a
direction for the finite difference which does not
violate any constraints.  The other methods do not
perform this check.  The two\sphinxhyphen{}sided method is in
principle more precise, but requires twice as many
function evaluations.  Default: 0.

\item[{‘mpmaxstep’ \sphinxhyphen{} the maximum change to be made in the parameter}] \leavevmode
value.  During the fitting process, the parameter
will never be changed by more than this value in
one iteration.

A value of 0 indicates no maximum.  Default: 0.

\item[{‘tied’ \sphinxhyphen{} a string expression which “ties” the parameter to other}] \leavevmode
free or fixed parameters.  Any expression involving
constants and the parameter array P are permitted.
Example: if parameter 2 is always to be twice parameter
1 then use the following: parinfo(2).tied = ‘2 * p(1)’.
Since they are totally constrained, tied parameters are
considered to be fixed; no errors are computed for them.
{[} NOTE: the PARNAME can’t be used in expressions. {]}

\item[{‘mpprint’ \sphinxhyphen{} if set to 1, then the default iterfunct will print the}] \leavevmode
parameter value.  If set to 0, the parameter value
will not be printed.  This tag can be used to
selectively print only a few parameter values out of
many.  Default: 1 (all parameters printed)

\end{description}
\end{quote}

Future modifications to the PARINFO structure, if any, will involve
adding dictionary tags beginning with the two letters “MP”.
Therefore programmers are urged to avoid using tags starting with
the same letters; otherwise they are free to include their own
fields within the PARINFO structure, and they will be ignored.

PARINFO Example:
parinfo = {[}\{‘value’:0., ‘fixed’:0, ‘limited’:{[}0,0{]}, ‘limits’:{[}0.,0.{]}\}
\begin{quote}

for i in range(5){]}
\end{quote}

parinfo{[}0{]}{[}‘fixed’{]} = 1
parinfo{[}4{]}{[}‘limited’{]}{[}0{]} = 1
parinfo{[}4{]}{[}‘limits’{]}{[}0{]}  = 50.
values = {[}5.7, 2.2, 500., 1.5, 2000.{]}
for i in range(5): parinfo{[}i{]}{[}‘value’{]}=values{[}i{]}

A total of 5 parameters, with starting values of 5.7,
2.2, 500, 1.5, and 2000 are given.  The first parameter
is fixed at a value of 5.7, and the last parameter is
constrained to be above 50.
\begin{quote}
\begin{quote}

EXAMPLE
\end{quote}

import mpfit
import numpy.oldnumeric as Numeric
x = arange(100, float)
p0 = {[}5.7, 2.2, 500., 1.5, 2000.{]}
y = ( p{[}0{]} + p{[}1{]}*{[}x{]} + p{[}2{]}*{[}x**2{]} + p{[}3{]}*sqrt(x) +
\begin{quote}

p{[}4{]}*log(x))
\end{quote}

fa = \{‘x’:x, ‘y’:y, ‘err’:err\}
m = mpfit(‘myfunct’, p0, functkw=fa)
print ‘status = ‘, m.status
if (m.status \textless{}= 0): print ‘error message = ‘, m.errmsg
print ‘parameters = ‘, m.params

Minimizes sum of squares of MYFUNCT.  MYFUNCT is called with the X,
Y, and ERR keyword parameters that are given by FUNCTKW.  The
results can be obtained from the returned object m.
\begin{quote}

THEORY OF OPERATION
\end{quote}

There are many specific strategies for function minimization.  One
very popular technique is to use function gradient information to
realize the local structure of the function.  Near a local minimum
the function value can be taylor expanded about x0 as follows:
\begin{quote}
\begin{quote}
\begin{description}
\item[{f(x) = f(x0) + f’(x0) . (x\sphinxhyphen{}x0) + (1/2) (x\sphinxhyphen{}x0) . f’’(x0) . (x\sphinxhyphen{}x0)}] \leavevmode
—\textendash{}   —————   ——————————\sphinxhyphen{}  (1)

\end{description}
\end{quote}

Order  0th               1st                                     2nd
\end{quote}

Here f’(x) is the gradient vector of f at x, and f’’(x) is the
Hessian matrix of second derivatives of f at x.  The vector x is
the set of function parameters, not the measured data vector.  One
can find the minimum of f, f(xm) using Newton’s method, and
arrives at the following linear equation:
\begin{quote}

f’’(x0) . (xm\sphinxhyphen{}x0) = \sphinxhyphen{} f’(x0)                                                  (2)
\end{quote}

If an inverse can be found for f’’(x0) then one can solve for
(xm\sphinxhyphen{}x0), the step vector from the current position x0 to the new
projected minimum.  Here the problem has been linearized (ie, the
gradient information is known to first order).  f’’(x0) is
symmetric n x n matrix, and should be positive definite.

The Levenberg \sphinxhyphen{} Marquardt technique is a variation on this theme.
It adds an additional diagonal term to the equation which may aid the
convergence properties:
\begin{quote}

(f’’(x0) + nu I) . (xm\sphinxhyphen{}x0) = \sphinxhyphen{}f’(x0)                            (2a)
\end{quote}

where I is the identity matrix.  When nu is large, the overall
matrix is diagonally dominant, and the iterations follow steepest
descent.  When nu is small, the iterations are quadratically
convergent.

In principle, if f’’(x0) and f’(x0) are known then xm\sphinxhyphen{}x0 can be
determined.  However the Hessian matrix is often difficult or
impossible to compute.  The gradient f’(x0) may be easier to
compute, if even by finite difference techniques.  So\sphinxhyphen{}called
quasi\sphinxhyphen{}Newton techniques attempt to successively estimate f’’(x0)
by building up gradient information as the iterations proceed.

In the least squares problem there are further simplifications
which assist in solving eqn (2).  The function to be minimized is
a sum of squares:
\begin{quote}

f = Sum(hi\textasciicircum{}2)                                                                                 (3)
\end{quote}

where hi is the ith residual out of m residuals as described
above.  This can be substituted back into eqn (2) after computing
the derivatives:
\begin{quote}

f’  = 2 Sum(hi  hi’)
f’’ = 2 Sum(hi’ hj’) + 2 Sum(hi hi’’)                                (4)
\end{quote}

If one assumes that the parameters are already close enough to a
minimum, then one typically finds that the second term in f’’ is
negligible {[}or, in any case, is too difficult to compute{]}.  Thus,
equation (2) can be solved, at least approximately, using only
gradient information.

In matrix notation, the combination of eqns (2) and (4) becomes:
\begin{quote}

hT’ . h’ . dx = \sphinxhyphen{} hT’ . h                                                 (5)
\end{quote}

Where h is the residual vector (length m), hT is its transpose, h’
is the Jacobian matrix (dimensions n x m), and dx is (xm\sphinxhyphen{}x0).  The
user function supplies the residual vector h, and in some cases h’
when it is not found by finite differences (see MPFIT\_FDJAC2,
which finds h and hT’).  Even if dx is not the best absolute step
to take, it does provide a good estimate of the best \sphinxstyleemphasis{direction},
so often a line minimization will occur along the dx vector
direction.

The method of solution employed by MINPACK is to form the Q . R
factorization of h’, where Q is an orthogonal matrix such that QT .
Q = I, and R is upper right triangular.  Using h’ = Q . R and the
ortogonality of Q, eqn (5) becomes
\begin{quote}
\begin{description}
\item[{(RT . QT) . (Q . R) . dx = \sphinxhyphen{} (RT . QT) . h}] \leavevmode\begin{description}
\item[{RT . R . dx = \sphinxhyphen{} RT . QT . h             (6)}] \leavevmode
R . dx = \sphinxhyphen{} QT . h

\end{description}

\end{description}
\end{quote}

where the last statement follows because R is upper triangular.
Here, R, QT and h are known so this is a matter of solving for dx.
The routine MPFIT\_QRFAC provides the QR factorization of h, with
pivoting, and MPFIT\_QRSOLV provides the solution for dx.
\begin{quote}

REFERENCES
\end{quote}

MINPACK\sphinxhyphen{}1, Jorge More’, available from netlib (www.netlib.org).
“Optimization Software Guide,” Jorge More’ and Stephen Wright,
\begin{quote}

SIAM, \sphinxstyleemphasis{Frontiers in Applied Mathematics}, Number 14.
\end{quote}
\begin{description}
\item[{More’, Jorge J., “The Levenberg\sphinxhyphen{}Marquardt Algorithm:}] \leavevmode
Implementation and Theory,” in \sphinxstyleemphasis{Numerical Analysis}, ed. Watson,
G. A., Lecture Notes in Mathematics 630, Springer\sphinxhyphen{}Verlag, 1977.
\begin{quote}

MODIFICATION HISTORY
\end{quote}

\end{description}

Translated from MINPACK\sphinxhyphen{}1 in FORTRAN, Apr\sphinxhyphen{}Jul 1998, CM
\end{quote}

Copyright (C) 1997\sphinxhyphen{}2002, Craig Markwardt
This software is provided as is without any warranty whatsoever.
Permission to use, copy, modify, and distribute modified or
unmodified copies is granted, provided this copyright and disclaimer
are included unchanged.
\begin{quote}

Translated from MPFIT (Craig Markwardt’s IDL package) to Python,
August, 2002.  Mark Rivers
Converted from Numeric to numpy (Sergey Koposov, July 2008)
\end{quote}
\end{quote}
\index{machar (class in FASMA.mpfit)@\spxentry{machar}\spxextra{class in FASMA.mpfit}}

\begin{fulllineitems}
\phantomsection\label{\detokenize{index:FASMA.mpfit.machar}}\pysiglinewithargsret{\sphinxbfcode{\sphinxupquote{class }}\sphinxcode{\sphinxupquote{FASMA.mpfit.}}\sphinxbfcode{\sphinxupquote{machar}}}{\emph{double=1}}{}
Bases: \sphinxcode{\sphinxupquote{object}}

\end{fulllineitems}

\index{mpfit (class in FASMA.mpfit)@\spxentry{mpfit}\spxextra{class in FASMA.mpfit}}

\begin{fulllineitems}
\phantomsection\label{\detokenize{index:FASMA.mpfit.mpfit}}\pysiglinewithargsret{\sphinxbfcode{\sphinxupquote{class }}\sphinxcode{\sphinxupquote{FASMA.mpfit.}}\sphinxbfcode{\sphinxupquote{mpfit}}}{\emph{fcn}, \emph{xall=None}, \emph{functkw=\{\}}, \emph{parinfo=None}, \emph{ftol=1e\sphinxhyphen{}10}, \emph{xtol=1e\sphinxhyphen{}10}, \emph{gtol=1e\sphinxhyphen{}10}, \emph{damp=0.0}, \emph{maxiter=200}, \emph{factor=100.0}, \emph{nprint=1}, \emph{iterfunct=\textquotesingle{}default\textquotesingle{}}, \emph{iterkw=\{\}}, \emph{nocovar=0}, \emph{rescale=0}, \emph{autoderivative=1}, \emph{quiet=0}, \emph{diag=None}, \emph{epsfcn=None}, \emph{debug=0}}{}
Bases: \sphinxcode{\sphinxupquote{object}}
\index{blas\_enorm32 (FASMA.mpfit.mpfit attribute)@\spxentry{blas\_enorm32}\spxextra{FASMA.mpfit.mpfit attribute}}

\begin{fulllineitems}
\phantomsection\label{\detokenize{index:FASMA.mpfit.mpfit.blas_enorm32}}\pysigline{\sphinxbfcode{\sphinxupquote{blas\_enorm32}}\sphinxbfcode{\sphinxupquote{ = \textless{}fortran dnrm2\textgreater{}}}}
\end{fulllineitems}

\index{blas\_enorm64 (FASMA.mpfit.mpfit attribute)@\spxentry{blas\_enorm64}\spxextra{FASMA.mpfit.mpfit attribute}}

\begin{fulllineitems}
\phantomsection\label{\detokenize{index:FASMA.mpfit.mpfit.blas_enorm64}}\pysigline{\sphinxbfcode{\sphinxupquote{blas\_enorm64}}\sphinxbfcode{\sphinxupquote{ = \textless{}fortran dnrm2\textgreater{}}}}
\end{fulllineitems}

\index{calc\_covar() (FASMA.mpfit.mpfit method)@\spxentry{calc\_covar()}\spxextra{FASMA.mpfit.mpfit method}}

\begin{fulllineitems}
\phantomsection\label{\detokenize{index:FASMA.mpfit.mpfit.calc_covar}}\pysiglinewithargsret{\sphinxbfcode{\sphinxupquote{calc\_covar}}}{\emph{rr}, \emph{ipvt=None}, \emph{tol=1e\sphinxhyphen{}14}}{}
\end{fulllineitems}

\index{call() (FASMA.mpfit.mpfit method)@\spxentry{call()}\spxextra{FASMA.mpfit.mpfit method}}

\begin{fulllineitems}
\phantomsection\label{\detokenize{index:FASMA.mpfit.mpfit.call}}\pysiglinewithargsret{\sphinxbfcode{\sphinxupquote{call}}}{\emph{fcn}, \emph{x}, \emph{functkw}, \emph{fjac=None}}{}
\end{fulllineitems}

\index{defiter() (FASMA.mpfit.mpfit method)@\spxentry{defiter()}\spxextra{FASMA.mpfit.mpfit method}}

\begin{fulllineitems}
\phantomsection\label{\detokenize{index:FASMA.mpfit.mpfit.defiter}}\pysiglinewithargsret{\sphinxbfcode{\sphinxupquote{defiter}}}{\emph{fcn}, \emph{x}, \emph{iter}, \emph{fnorm=None}, \emph{functkw=None}, \emph{quiet=0}, \emph{iterstop=None}, \emph{parinfo=None}, \emph{format=None}, \emph{pformat=\textquotesingle{}\%.10g\textquotesingle{}}, \emph{dof=1}}{}
\end{fulllineitems}

\index{enorm() (FASMA.mpfit.mpfit method)@\spxentry{enorm()}\spxextra{FASMA.mpfit.mpfit method}}

\begin{fulllineitems}
\phantomsection\label{\detokenize{index:FASMA.mpfit.mpfit.enorm}}\pysiglinewithargsret{\sphinxbfcode{\sphinxupquote{enorm}}}{\emph{vec}}{}
\end{fulllineitems}

\index{fdjac2() (FASMA.mpfit.mpfit method)@\spxentry{fdjac2()}\spxextra{FASMA.mpfit.mpfit method}}

\begin{fulllineitems}
\phantomsection\label{\detokenize{index:FASMA.mpfit.mpfit.fdjac2}}\pysiglinewithargsret{\sphinxbfcode{\sphinxupquote{fdjac2}}}{\emph{fcn}, \emph{x}, \emph{fvec}, \emph{step=None}, \emph{ulimited=None}, \emph{ulimit=None}, \emph{dside=None}, \emph{epsfcn=None}, \emph{autoderivative=1}, \emph{functkw=None}, \emph{xall=None}, \emph{ifree=None}, \emph{dstep=None}}{}
\end{fulllineitems}

\index{lmpar() (FASMA.mpfit.mpfit method)@\spxentry{lmpar()}\spxextra{FASMA.mpfit.mpfit method}}

\begin{fulllineitems}
\phantomsection\label{\detokenize{index:FASMA.mpfit.mpfit.lmpar}}\pysiglinewithargsret{\sphinxbfcode{\sphinxupquote{lmpar}}}{\emph{r}, \emph{ipvt}, \emph{diag}, \emph{qtb}, \emph{delta}, \emph{x}, \emph{sdiag}, \emph{par=None}}{}
\end{fulllineitems}

\index{parinfo() (FASMA.mpfit.mpfit method)@\spxentry{parinfo()}\spxextra{FASMA.mpfit.mpfit method}}

\begin{fulllineitems}
\phantomsection\label{\detokenize{index:FASMA.mpfit.mpfit.parinfo}}\pysiglinewithargsret{\sphinxbfcode{\sphinxupquote{parinfo}}}{\emph{parinfo=None}, \emph{key=\textquotesingle{}a\textquotesingle{}}, \emph{default=None}, \emph{n=0}}{}
\end{fulllineitems}

\index{qrfac() (FASMA.mpfit.mpfit method)@\spxentry{qrfac()}\spxextra{FASMA.mpfit.mpfit method}}

\begin{fulllineitems}
\phantomsection\label{\detokenize{index:FASMA.mpfit.mpfit.qrfac}}\pysiglinewithargsret{\sphinxbfcode{\sphinxupquote{qrfac}}}{\emph{a}, \emph{pivot=0}}{}
\end{fulllineitems}

\index{qrsolv() (FASMA.mpfit.mpfit method)@\spxentry{qrsolv()}\spxextra{FASMA.mpfit.mpfit method}}

\begin{fulllineitems}
\phantomsection\label{\detokenize{index:FASMA.mpfit.mpfit.qrsolv}}\pysiglinewithargsret{\sphinxbfcode{\sphinxupquote{qrsolv}}}{\emph{r}, \emph{ipvt}, \emph{diag}, \emph{qtb}, \emph{sdiag}}{}
\end{fulllineitems}

\index{tie() (FASMA.mpfit.mpfit method)@\spxentry{tie()}\spxextra{FASMA.mpfit.mpfit method}}

\begin{fulllineitems}
\phantomsection\label{\detokenize{index:FASMA.mpfit.mpfit.tie}}\pysiglinewithargsret{\sphinxbfcode{\sphinxupquote{tie}}}{\emph{p}, \emph{ptied=None}}{}
\end{fulllineitems}


\end{fulllineitems}



\section{FASMA.observations module}
\label{\detokenize{index:module-FASMA.observations}}\label{\detokenize{index:fasma-observations-module}}\index{FASMA.observations (module)@\spxentry{FASMA.observations}\spxextra{module}}\index{eso\_fits() (in module FASMA.observations)@\spxentry{eso\_fits()}\spxextra{in module FASMA.observations}}

\begin{fulllineitems}
\phantomsection\label{\detokenize{index:FASMA.observations.eso_fits}}\pysiglinewithargsret{\sphinxcode{\sphinxupquote{FASMA.observations.}}\sphinxbfcode{\sphinxupquote{eso\_fits}}}{\emph{hdulist}}{}
A little demo utility to illustrate the ESO SDP 1D spectrum file format.
2017\sphinxhyphen{}Aug\sphinxhyphen{}08, archive(at)eso.org
The ESO 1D spectral format is compliant with the IVOA Spectrum data model.
Both the SPECTRUM 1.0 and SPECTRUM 2.0 versions are supported.
References:
ESO SDP standard: \sphinxurl{http://www.eso.org/sci/observing/phase3/p3sdpstd.pdf}
ESO SDP FAQ page: \sphinxurl{http://www.eso.org/sci/observing/phase3/faq.html}
This script:      \sphinxurl{http://archive.eso.org/cms/eso-data/help/1dspectra.html}

wave : raw observed wavelength
flux : raw observed flux

\end{fulllineitems}

\index{local\_norm() (in module FASMA.observations)@\spxentry{local\_norm()}\spxextra{in module FASMA.observations}}

\begin{fulllineitems}
\phantomsection\label{\detokenize{index:FASMA.observations.local_norm}}\pysiglinewithargsret{\sphinxcode{\sphinxupquote{FASMA.observations.}}\sphinxbfcode{\sphinxupquote{local\_norm}}}{\emph{obs\_fname}, \emph{r}, \emph{snr}, \emph{lol=1.0}, \emph{plot=False}}{}
Very local Normalization function. Makes a linear fit from the maximum points
of each segment.
Input
—\textendash{}
obs\_fname : observations file
r : range of the interval
plot: True to visual check if normalization is correct, default: False

wave : wavelength
new\_flux : normalized flux
delta\_l : wavelenghth spacing

\end{fulllineitems}

\index{mad() (in module FASMA.observations)@\spxentry{mad()}\spxextra{in module FASMA.observations}}

\begin{fulllineitems}
\phantomsection\label{\detokenize{index:FASMA.observations.mad}}\pysiglinewithargsret{\sphinxcode{\sphinxupquote{FASMA.observations.}}\sphinxbfcode{\sphinxupquote{mad}}}{\emph{data}, \emph{axis=None}}{}
Function to calculate the median average deviation.

\end{fulllineitems}

\index{plot() (in module FASMA.observations)@\spxentry{plot()}\spxextra{in module FASMA.observations}}

\begin{fulllineitems}
\phantomsection\label{\detokenize{index:FASMA.observations.plot}}\pysiglinewithargsret{\sphinxcode{\sphinxupquote{FASMA.observations.}}\sphinxbfcode{\sphinxupquote{plot}}}{\emph{xobs}, \emph{yobs}, \emph{xinit}, \emph{yinit}, \emph{xfinal}, \emph{yfinal}, \emph{res=False}}{}
Function to plot synthetic and observed spectra.
Input
—\textendash{}
xobs : observed wavelength
yobs : observed flux
xinit : synthetic wavelength with initial parameters
yinit : synthetic flux with initial parameters
xfinal : synthetic wavelength with final parameters
yfinal : synthetic flux with final parameters
res: Flag to plot residuals

plots

\end{fulllineitems}

\index{read\_obs\_intervals() (in module FASMA.observations)@\spxentry{read\_obs\_intervals()}\spxextra{in module FASMA.observations}}

\begin{fulllineitems}
\phantomsection\label{\detokenize{index:FASMA.observations.read_obs_intervals}}\pysiglinewithargsret{\sphinxcode{\sphinxupquote{FASMA.observations.}}\sphinxbfcode{\sphinxupquote{read\_obs\_intervals}}}{\emph{obs\_fname}, \emph{r}, \emph{snr=None}}{}
Read only the spectral chunks from the observed spectrum and normalize
the regions.
Input
—\textendash{}
fname : filename of the spectrum.
r : ranges of wavelength intervals where the observed spectrum is cut
(starting and ending wavelength)
snr: signal\sphinxhyphen{}to\sphinxhyphen{}noise

xobs : observed normalized wavelength
yobs : observed normalized flux
delta\_l : wavelenghth spacing

\end{fulllineitems}

\index{read\_observations() (in module FASMA.observations)@\spxentry{read\_observations()}\spxextra{in module FASMA.observations}}

\begin{fulllineitems}
\phantomsection\label{\detokenize{index:FASMA.observations.read_observations}}\pysiglinewithargsret{\sphinxcode{\sphinxupquote{FASMA.observations.}}\sphinxbfcode{\sphinxupquote{read\_observations}}}{\emph{fname}, \emph{start\_synth}, \emph{end\_synth}}{}
Read observed spectrum of different types and return wavelength and flux.
Input
—\textendash{}
fname : filename of the spectrum. These are the approved formats: ‘.dat’, ‘.txt’,
‘.spec’, ‘.fits’.
start\_synth : starting wavelength where the observed spectrum is cut
end\_synth : ending wavelength where the observed spectrum is cut

wave\_obs : raw observed wavelength
flux\_obs : raw observed flux

\end{fulllineitems}

\index{snr() (in module FASMA.observations)@\spxentry{snr()}\spxextra{in module FASMA.observations}}

\begin{fulllineitems}
\phantomsection\label{\detokenize{index:FASMA.observations.snr}}\pysiglinewithargsret{\sphinxcode{\sphinxupquote{FASMA.observations.}}\sphinxbfcode{\sphinxupquote{snr}}}{\emph{fname}, \emph{plot=False}}{}
Calculate SNR using for various intervals.
Input
—\sphinxhyphen{}
fname : spectrum
plot : plot snr fit

snr : snr value averaged from the continuum intervals

\end{fulllineitems}



\section{FASMA.solar\_abundance module}
\label{\detokenize{index:module-FASMA.solar_abundance}}\label{\detokenize{index:fasma-solar-abundance-module}}\index{FASMA.solar\_abundance (module)@\spxentry{FASMA.solar\_abundance}\spxextra{module}}

\section{FASMA.synthDriver module}
\label{\detokenize{index:module-FASMA.synthDriver}}\label{\detokenize{index:fasma-synthdriver-module}}\index{FASMA.synthDriver (module)@\spxentry{FASMA.synthDriver}\spxextra{module}}\index{FASMA (class in FASMA.synthDriver)@\spxentry{FASMA}\spxextra{class in FASMA.synthDriver}}

\begin{fulllineitems}
\phantomsection\label{\detokenize{index:FASMA.synthDriver.FASMA}}\pysiglinewithargsret{\sphinxbfcode{\sphinxupquote{class }}\sphinxcode{\sphinxupquote{FASMA.synthDriver.}}\sphinxbfcode{\sphinxupquote{FASMA}}}{\emph{cfgfile=\textquotesingle{}config.yml\textquotesingle{}}, \emph{overwrite=None}, \emph{**kwargs}}{}
Bases: \sphinxcode{\sphinxupquote{object}}
\index{configure() (FASMA.synthDriver.FASMA method)@\spxentry{configure()}\spxextra{FASMA.synthDriver.FASMA method}}

\begin{fulllineitems}
\phantomsection\label{\detokenize{index:FASMA.synthDriver.FASMA.configure}}\pysiglinewithargsret{\sphinxbfcode{\sphinxupquote{configure}}}{\emph{cfgfile=\textquotesingle{}config.yml\textquotesingle{}}, \emph{**kwargs}}{}
Create configuration file from kwargs.
Otherwise set to default.

\end{fulllineitems}

\index{minizationElementRunner() (FASMA.synthDriver.FASMA method)@\spxentry{minizationElementRunner()}\spxextra{FASMA.synthDriver.FASMA method}}

\begin{fulllineitems}
\phantomsection\label{\detokenize{index:FASMA.synthDriver.FASMA.minizationElementRunner}}\pysiglinewithargsret{\sphinxbfcode{\sphinxupquote{minizationElementRunner}}}{\emph{p=None}}{}
A function to run the minimization routine for element abundances.

params : output parameters

\end{fulllineitems}

\index{minizationRunner() (FASMA.synthDriver.FASMA method)@\spxentry{minizationRunner()}\spxextra{FASMA.synthDriver.FASMA method}}

\begin{fulllineitems}
\phantomsection\label{\detokenize{index:FASMA.synthDriver.FASMA.minizationRunner}}\pysiglinewithargsret{\sphinxbfcode{\sphinxupquote{minizationRunner}}}{\emph{p=None}}{}
A function to run the minimization routine

params : output parameters

\end{fulllineitems}

\index{plotRunner() (FASMA.synthDriver.FASMA method)@\spxentry{plotRunner()}\spxextra{FASMA.synthDriver.FASMA method}}

\begin{fulllineitems}
\phantomsection\label{\detokenize{index:FASMA.synthDriver.FASMA.plotRunner}}\pysiglinewithargsret{\sphinxbfcode{\sphinxupquote{plotRunner}}}{\emph{x=None}, \emph{y=None}, \emph{xs=None}, \emph{ys=None}, \emph{xf=None}, \emph{yf=None}, \emph{res=False}}{}
A function to plot spectra with or without residuals.

plots

\end{fulllineitems}

\index{result() (FASMA.synthDriver.FASMA method)@\spxentry{result()}\spxextra{FASMA.synthDriver.FASMA method}}

\begin{fulllineitems}
\phantomsection\label{\detokenize{index:FASMA.synthDriver.FASMA.result}}\pysiglinewithargsret{\sphinxbfcode{\sphinxupquote{result}}}{}{}
If any, get the output parameters.

status : if None, no minimization happened

params : output parameters in dictionary

\end{fulllineitems}

\index{saveRunner() (FASMA.synthDriver.FASMA method)@\spxentry{saveRunner()}\spxextra{FASMA.synthDriver.FASMA method}}

\begin{fulllineitems}
\phantomsection\label{\detokenize{index:FASMA.synthDriver.FASMA.saveRunner}}\pysiglinewithargsret{\sphinxbfcode{\sphinxupquote{saveRunner}}}{}{}
A function to save spectra in a fits like format in the results/ folder.
If initial spectrum exists, save it.

saved spectrum

\end{fulllineitems}

\index{synthdriver() (FASMA.synthDriver.FASMA method)@\spxentry{synthdriver()}\spxextra{FASMA.synthDriver.FASMA method}}

\begin{fulllineitems}
\phantomsection\label{\detokenize{index:FASMA.synthDriver.FASMA.synthdriver}}\pysiglinewithargsret{\sphinxbfcode{\sphinxupquote{synthdriver}}}{}{}
A function to connect all. This function applies the options set by
the user in the config.yml file.

\end{fulllineitems}


\end{fulllineitems}



\section{FASMA.synthetic module}
\label{\detokenize{index:module-FASMA.synthetic}}\label{\detokenize{index:fasma-synthetic-module}}\index{FASMA.synthetic (module)@\spxentry{FASMA.synthetic}\spxextra{module}}\index{broadening() (in module FASMA.synthetic)@\spxentry{broadening()}\spxextra{in module FASMA.synthetic}}

\begin{fulllineitems}
\phantomsection\label{\detokenize{index:FASMA.synthetic.broadening}}\pysiglinewithargsret{\sphinxcode{\sphinxupquote{FASMA.synthetic.}}\sphinxbfcode{\sphinxupquote{broadening}}}{\emph{x}, \emph{y}, \emph{vsini}, \emph{vmac}, \emph{resolution=None}, \emph{epsilon=0.6}}{}
This function broadens the given data using velocity kernels,
e.g. instrumental profile, vsini and vmac.
Based on \sphinxurl{http://www.hs.uni-hamburg.de/DE/Ins/Per/Czesla/PyA/PyA/pyaslDoc/aslDoc/broadening.html}
Input
—\sphinxhyphen{}
x : ndarray
\begin{quote}

wavelength
\end{quote}
\begin{description}
\item[{y}] \leavevmode{[}ndarray{]}
flux

\item[{resolution}] \leavevmode{[}float{]}
Instrumental resolution (lambda /delta lambda)

\item[{vsini}] \leavevmode{[}float{]}
vsini in km/s

\item[{vmac}] \leavevmode{[}float{]}
vmac in km/s

\end{description}

epsilon : limb\sphinxhyphen{}darkening parameter
\begin{description}
\item[{y\_broad}] \leavevmode{[}ndarray{]}
Broadened flux

\item[{x}] \leavevmode{[}ndarray{]}
Same wavelength

\end{description}

\end{fulllineitems}

\index{read\_linelist() (in module FASMA.synthetic)@\spxentry{read\_linelist()}\spxextra{in module FASMA.synthetic}}

\begin{fulllineitems}
\phantomsection\label{\detokenize{index:FASMA.synthetic.read_linelist}}\pysiglinewithargsret{\sphinxcode{\sphinxupquote{FASMA.synthetic.}}\sphinxbfcode{\sphinxupquote{read\_linelist}}}{\emph{fname}, \emph{intname=\textquotesingle{}intervals.lst\textquotesingle{}}}{}
Read the line list (atomic data) and the file which includes the ranges
where the synthesis will happen.
\begin{description}
\item[{fname}] \leavevmode{[}str{]}
File that contains the linelist

\item[{intname}] \leavevmode{[}str{]}
File that contains the intervals

\end{description}

ranges : wavelength ranges of the linelist
atomic : atomic data

\end{fulllineitems}

\index{read\_linelist\_elem() (in module FASMA.synthetic)@\spxentry{read\_linelist\_elem()}\spxextra{in module FASMA.synthetic}}

\begin{fulllineitems}
\phantomsection\label{\detokenize{index:FASMA.synthetic.read_linelist_elem}}\pysiglinewithargsret{\sphinxcode{\sphinxupquote{FASMA.synthetic.}}\sphinxbfcode{\sphinxupquote{read\_linelist\_elem}}}{\emph{fname}, \emph{element=None}, \emph{intname=\textquotesingle{}intervals\_elements.lst\textquotesingle{}}}{}
Read the line list (atomic data) and the file which includes the ranges
where the synthesis will happen for the element abundances.
\begin{description}
\item[{fname}] \leavevmode{[}str{]}
File that contains the linelist

\item[{element}] \leavevmode{[}str{]}
The element to be searched in the line list

\item[{intname}] \leavevmode{[}str{]}
File that contains the central line where \sphinxhyphen{}+2.0 AA\{\} are added to create
the interval around each line.

\end{description}

ranges : wavelength ranges of the linelist
atomic : atomic data

\end{fulllineitems}

\index{save\_synth\_spec() (in module FASMA.synthetic)@\spxentry{save\_synth\_spec()}\spxextra{in module FASMA.synthetic}}

\begin{fulllineitems}
\phantomsection\label{\detokenize{index:FASMA.synthetic.save_synth_spec}}\pysiglinewithargsret{\sphinxcode{\sphinxupquote{FASMA.synthetic.}}\sphinxbfcode{\sphinxupquote{save\_synth\_spec}}}{\emph{x}, \emph{y}, \emph{initial=None}, \emph{**options}}{}
Save synthetic spectrum of all intervals
\begin{description}
\item[{x}] \leavevmode{[}ndarray{]}
Wavelength

\item[{y}] \leavevmode{[}ndarray{]}
Flux

\item[{initial}] \leavevmode{[}list{]}
Set of parameters to name the new file, else it is named ‘synthetic.spec’.

\end{description}

fname fits file

\end{fulllineitems}



\section{FASMA.utils module}
\label{\detokenize{index:module-FASMA.utils}}\label{\detokenize{index:fasma-utils-module}}\index{FASMA.utils (module)@\spxentry{FASMA.utils}\spxextra{module}}\index{GetModels (class in FASMA.utils)@\spxentry{GetModels}\spxextra{class in FASMA.utils}}

\begin{fulllineitems}
\phantomsection\label{\detokenize{index:FASMA.utils.GetModels}}\pysiglinewithargsret{\sphinxbfcode{\sphinxupquote{class }}\sphinxcode{\sphinxupquote{FASMA.utils.}}\sphinxbfcode{\sphinxupquote{GetModels}}}{\emph{teff}, \emph{logg}, \emph{feh}, \emph{atmtype}}{}
Bases: \sphinxcode{\sphinxupquote{object}}

Find the names of the closest grid points for a given effective
temperature, surface gravity, and iron abundance (proxy for metallicity).
\begin{description}
\item[{teff}] \leavevmode{[}int{]}
The effective temperature (K) for the model atmosphere

\item[{logg}] \leavevmode{[}float{]}
The surface gravity (logarithmic in cgs) for the model atmosphere

\item[{feh}] \leavevmode{[}float{]}
The metallicity for the model atmosphere

\item[{atmtype}] \leavevmode{[}str{]}
The type of atmosphere models to use. Currently only Kurucz from ‘95.

\end{description}
\index{getmodels() (FASMA.utils.GetModels method)@\spxentry{getmodels()}\spxextra{FASMA.utils.GetModels method}}

\begin{fulllineitems}
\phantomsection\label{\detokenize{index:FASMA.utils.GetModels.getmodels}}\pysiglinewithargsret{\sphinxbfcode{\sphinxupquote{getmodels}}}{}{}
Get the atmosphere models surrounding the requested atmospheric
parameters. This function should be called before using the interpolation
code within FASMA.
\begin{description}
\item[{models}] \leavevmode{[}list{]}
List with path to 8 models the two closest in each parameter space (4x2x2)

\item[{teff\_model}] \leavevmode{[}list{]}
The four closest effective temperatures in the grid

\item[{logg\_model}] \leavevmode{[}list{]}
The two closest surface gravities in the grid

\item[{feh\_model}] \leavevmode{[}list{]}
The two closest metallicities in the grid

\end{description}

The last three return values are used for the interpolation to do some
mapping. If only the paths to the models are needed, do not pay attention
to them.

\end{fulllineitems}

\index{neighbour() (FASMA.utils.GetModels method)@\spxentry{neighbour()}\spxextra{FASMA.utils.GetModels method}}

\begin{fulllineitems}
\phantomsection\label{\detokenize{index:FASMA.utils.GetModels.neighbour}}\pysiglinewithargsret{\sphinxbfcode{\sphinxupquote{neighbour}}}{\emph{arr}, \emph{val}, \emph{k=2}}{}
Return the K surrounding neighbours of an array, given a certain value.
\begin{description}
\item[{arr}] \leavevmode{[}array\_like{]}
The array from which some neighbours should be found (assumed sorted).

\item[{val}] \leavevmode{[}float{]}
The value to found neighbours around in arr

\item[{k}] \leavevmode{[}int{]}
The number of neighbours to find

\end{description}
\begin{description}
\item[{array}] \leavevmode{[}list{]}
A list with the k surrounding neighbours

\end{description}

\end{fulllineitems}


\end{fulllineitems}

\index{fun\_moog\_synth() (in module FASMA.utils)@\spxentry{fun\_moog\_synth()}\spxextra{in module FASMA.utils}}

\begin{fulllineitems}
\phantomsection\label{\detokenize{index:FASMA.utils.fun_moog_synth}}\pysiglinewithargsret{\sphinxcode{\sphinxupquote{FASMA.utils.}}\sphinxbfcode{\sphinxupquote{fun\_moog\_synth}}}{\emph{x}, \emph{atmtype}, \emph{abund=0.0}, \emph{par=\textquotesingle{}batch.par\textquotesingle{}}, \emph{ranges=None}, \emph{results=\textquotesingle{}summary.out\textquotesingle{}}, \emph{driver=\textquotesingle{}synth\textquotesingle{}}, \emph{version=2014}, \emph{**options}}{}
Run MOOG and create synthetic spectrum for the synth driver.
:x: A tuple/list with values (teff, logg, {[}Fe/H{]}, vt, vmic, vmac)
:atmtype: model atmosphere
:abund: abundance of specific species
:ranges: array of intervals for the synthesis
:par: The parameter file (batch.par)
:results: The summary file
:driver: Which driver to use when running MOOG
:version: The version of MOOG
:returns: w, f : wavelength and flux

\end{fulllineitems}



\renewcommand{\indexname}{Python Module Index}
\begin{sphinxtheindex}
\let\bigletter\sphinxstyleindexlettergroup
\bigletter{f}
\item\relax\sphinxstyleindexentry{FASMA.tests}\sphinxstyleindexpageref{FASMA.tests:\detokenize{module-FASMA.tests}}
\item\relax\sphinxstyleindexentry{FASMA.tests.test\_FASMA}\sphinxstyleindexpageref{FASMA.tests:\detokenize{module-FASMA.tests.test_FASMA}}
\end{sphinxtheindex}

\renewcommand{\indexname}{Index}
\printindex
\end{document}