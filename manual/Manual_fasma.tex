\documentclass[a4paper,12pt]{article}
%\usepackage[a4paper]{geometry}
\usepackage[top=1in, bottom=1.25in, left=0.5in, right=0.5in]{geometry}
\usepackage[utf8]{inputenc}
\usepackage{listings}
\usepackage{color}
\definecolor{backcolour}{rgb}{0.95,0.95,0.92}
\definecolor{codegreen}{rgb}{0,0.6,0}
\lstdefinestyle{mystyle}{
    backgroundcolor=\color{backcolour},
    keepspaces=true,
    showspaces=false,
    showstringspaces=false,
    showtabs=false,
    tabsize=2
}

\lstset{style=mystyle}
%opening
\title{FASMA tutorial}
\author{}

\begin{document}

\maketitle


\section{The configuration file}

To use FASMA in an automatic way, configure the ``StarMe\_synth.cfg'. This file should be in the folder where FASMA runs. There are various configurations depending on what the user wants. 
A mandatory requirement is the line list. 

\begin{itemize}
 \item To create a synthetic spectrum with solar values, only the line list file is required, e.g.:
 \begin{lstlisting}[language=Python]
 #linelist

 linelist_sun_arcturus_calib.lst
\end{lstlisting}
The complete path should be given for the list file in the configuration file (``StarMe\_synth.cfgs). The default line list is in the \texttt{rawLinelist} folder already provided within FASMA.
 \item For a given line list select your options. The options are space seperated from the line list file and the options are comma seperated, e.g.:
 {\footnotesize
 \begin{lstlisting}[language=Python]
 #linelist options

 linelist_sun_arcturus_calib.lst observations:Sun_HARPS.fits,resolution:115000,
 minimize,refine,inter_file:intervals.lst,plot
\end{lstlisting}}
This example indicates that for the observed spectrum 'Sun\_HARPS.fits' with resolution 115000, FASMA will minimize it starting from solar values using the refine option (see below). The 
spectral regions for the synthesis are difined in the intervals.lst file with the option inter\_file. The final spectrum will be plotted. 
For the not listed options, FASMA uses the default ones. {\bf The complete paths of the line list, spectrum, and intervals file should be given.} FASMA already includes the 
line list (linelist\_sun\_arcturus\_calib.lst), the interval file (intervals.lst) and a HARPS spectrum of the Sun (Sun\_HARPS.fits)

 \item For a given line list and a set of initial values, create a synthetic spectrum. The initial parameters are space seperated from the line list file, e.g.:
 \begin{lstlisting}[language=Python]
 #linelist teff logg [M/H] vt vmac vsini

 linelist_sun_arcturus_calib.lst 5777 4.44 0.0 1.0 3.21 1.9
\end{lstlisting}

 \item For a given line list and a set of initial values, select your options. {\bf The initial parameters are space seperated from the line list file and the options are comma seperated}, e.g.:
 {\tiny
 \begin{lstlisting}[language=Python]
 #linelist teff logg [M/H] vt vmac vsini options

 linelist_sun_arcturus_calib.lst 5777 4.44 0.0 1.0 3.21 1.9 observations:Sun_HARPS.fits,resolution:115000,
 minimize,vt,vmac,refine
\end{lstlisting} }
The latter configuration is the suggested methodology (Tsantaki et al. 2018). Microturbulence (vt) and macroturbulence (vmac) are set fixed. 

\end{itemize}


\subsection{Default options}

The default options of FASMA can be changed in the configuration file `StarMe\_synth.cfg`.
\begin{enumerate}
 \item
\begin{lstlisting}[language=Python]
spt:          False

\end{lstlisting}
% \\*
In case the initial parameters of the stars are not known, the user can input the spectral type and luminosity class as shown in the SpectralTypes.yml file, e.g. spt:G2V.

\item
\begin{lstlisting}[language=Python]
model:        marcs
\end{lstlisting}
% \\*
The model atmospheres included in FASMA are: apogee\_kurucz, and marcs.

\item
\begin{lstlisting}[language=Python]
MOOGv:        2017
\end{lstlisting}

% \\*
The version of MOOG currently is 2017 but can easily updated to newer versions. This is relevant only if the output files change for future updates of MOOG.

\item
\begin{lstlisting}[language=Python]
save:         False
\end{lstlisting}
% \\*
If save = True, the output synthetic spectrum is saved in the folder \texttt{results}. The output is saved in a .spec format and is read with the astropy.io fits module.

\item
\begin{lstlisting}[language=Python]
teff:         False
logg:         False
feh:          False
vt:           False
vmac:         False
vsini:        False
\end{lstlisting}
% \\*
If any of these options is set, then they are fixed in the minimization process. 

\item
\begin{lstlisting}[language=Python]
plot:         False
plot_res:     False
\end{lstlisting}
% \\*
These are plotting options of the synthetic/observed spectra (plot) and of their residuals in case of minimization (plot\_res).

\item
\begin{lstlisting}[language=Python]
damping:      1
\end{lstlisting}
% \\*
The damping option of MOOG to deal with the van der Waals broadening. The acceptable values are: 0, 1, 2.

\item
\begin{lstlisting}[language=Python]
step_wave:    0.01
step_flux:    3.0
\end{lstlisting}
% \\*
The wavelength step for the synthetic spectrum to be created and the flux limit to consider opacity distributions from the neighboring lines measured in \AA{}. These are the same options of MOOG.

\item
\begin{lstlisting}[language=Python]
minimize:     False
\end{lstlisting}
% \\*
If minimize appears in the options, the minimization process will start to find the best-fit parameters for the observed spectrum, therefore observations must be included in the options.

\item
\begin{lstlisting}[language=Python]
refine:       False
\end{lstlisting}
% \\*
The refine option performs corrections on the derived parameters after the minimization for a more detailed analysis. After a first run, a second minimization starts with optimized
microturbulence and macroturbulence values according to the parameters of the first run.

\item
\begin{lstlisting}[language=Python]
observations: False
\end{lstlisting}
% \\*
The name of the spectrum is given here. The full path of the spectrum should be given.

\item
\begin{lstlisting}[language=Python]
inter_file:   intervals.lst
\end{lstlisting}
% \\*
This is the file which contains the intervals of the synthesis. FASMA already has specific intervals which correspond to the given line list. This file should be inside the \texttt{rawLinelist} folder. 
The complete path should be given. 

\item
\begin{lstlisting}[language=Python]
snr:          None
\end{lstlisting}
% \\*
The signal-to-noise (SNR) values of the observed spectra can be given by the user (optional). They are used in the normalization process. Otherwise the SNR is calculated by the PyAstronomy function 
within FASMA.

\item
\begin{lstlisting}[language=Python]
resolution:   None
\end{lstlisting}
% \\*
The resolution of the synthetic spectrum or the resolution of the observed spectrum. This value is necessary when minimization is on.

\item
\begin{lstlisting}[language=Python]
limb:         0.6
\end{lstlisting}
% \\*
The limb darkening coefficient for the vsini calculations. It is set fixed to 0.6 but can have different values depending on luminosity class.

\item
\begin{lstlisting}[language=Python]
element:     False
\end{lstlisting}
% \\*
This option will start the minimization process to derive the chemical abundance of a specific element. The element has to be set from this list: 
Na, Mg, Al, Si, Ca, Sc, Ti, V, Cr, Mn, Ni. In this case the line list and intervals are different than the standard derivation of the stellar parameters. 
They are also included in FASMA, elements.lst and intervals\_elements.lst, respectively in the \texttt{rawLinelist} folder. For the derivation of the chemical 
abundance the minimize option should be False. 

\end{enumerate}


\section{Run FASMA}

There different ways to run FASMA. FASMA is installed as a module. One way is to open the GUI with simple running:
\begin{lstlisting}[language=Python]
fasma_gui
\end{lstlisting}
This will create the configuration file for you, but will only allow you to analyse one star.

To run the batch version, the ``StarMe\_synth.cfg' file has to configured and then simply run:
\begin{lstlisting}[language=Python]
fasma
\end{lstlisting}

If your configuration file is called something else (recommended for different projects), then run
\begin{lstlisting}[language=Python]
fasma myConf.cfg
\end{lstlisting}

The module can be imported in scripts as:
\begin{lstlisting}[language=Python]
import FASMA
\end{lstlisting}

\end{document}
