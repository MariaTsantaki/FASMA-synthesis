\documentclass[a4paper,10pt]{article}
%\usepackage[a4paper]{geometry}
\usepackage[top=1in, bottom=1.25in, left=0.5in, right=0.5in]{geometry}
\usepackage[utf8]{inputenc}
\usepackage{listings}
\usepackage{color}
\definecolor{backcolour}{rgb}{0.95,0.95,0.92}
\definecolor{codegreen}{rgb}{0,0.6,0}
\lstdefinestyle{mystyle}{
    backgroundcolor=\color{backcolour},
    keepspaces=true,
    showspaces=false,
    showstringspaces=false,
    showtabs=false,
    tabsize=2
}

\lstset{style=mystyle}
%opening
\title{FASMA tutorial}
\author{}

\begin{document}

\maketitle


\section{The configuration file}

To use FASMA in an automatic way, configure the ``StarMe\_synth.cfg'. There are four acceptable configurations. For more options, the synthDriver has to be adapted.

\begin{itemize}
 \item To create a synthetic spectrum with solar values, only the line list file is required, e.g.:
 \begin{lstlisting}[language=Python]
 #linelist

 giraffe_sun_arcturus_calib.lst
\end{lstlisting}
The line list file should be in the rawLinelist folder. A line list is given already with FASMA.
 \item For a given line list select your options. The options are space seperated from the line list file and the options are comma seperated, e.g.:
 {\footnotesize
 \begin{lstlisting}[language=Python]
 #linelist options

 giraffe_sun_arcturus_calib.lst observations:Sun_HARPS.fits,resolution:115000,minimize,refine
\end{lstlisting}}
This example means that for the observed spectrum 'Sun\_HARPS.fits' with resolution 115000, FASMA will minimize it starting from solar values using the refine option.
For the not listed options, FASMA uses the default ones.

 \item For a given line list and a set of initial values, create a synthetic spectrum. The initial parameters are space seperated from the line list file, e.g.:
 \begin{lstlisting}[language=Python]
 #linelist teff logg [M/H] vt vmac vsini

 giraffe_sun_arcturus_calib.lst 5777 4.44 0.0 1.0 3.21 1.9
\end{lstlisting}

 \item For a given line list and a set of initial values select your options. The initial parameters are space seperated from the line list file and the options are comma seperated, e.g.:
 {\tiny
 \begin{lstlisting}[language=Python]
 #linelist teff logg [M/H] vt vmac vsini options

 giraffe_sun_arcturus_calib.lst 5777 4.44 0.0 1.0 3.21 1.9 observations:Sun_HARPS.fits,resolution:115000,minimize,vt,vmac,refine
\end{lstlisting} }
The latter configuration is the suggested methodology.

\end{itemize}
% \\*
Once the configuration file is set, run FASMA:
\begin{lstlisting}[language=Python]
python synthDriver.py
\end{lstlisting}


\subsection{Default options}

The default options of FASMA can be changed in the configuration file `StarMe\_synth.cfg`.
\begin{enumerate}
 \item
\begin{lstlisting}[language=Python]
spt:          False

\end{lstlisting}
% \\*
In case the initial parameters of the stars are not known, the user can input the spectral type and luminosity class as shown in the SpectralTypes.yml file, e.g. spt:G2V.

\item
\begin{lstlisting}[language=Python]
model:        marcs
\end{lstlisting}
% \\*
The model atmospheres included in FASMA are: kurucz08, apogee\_kurucz, and marcs.

\item
\begin{lstlisting}[language=Python]
MOOGv:        2014
\end{lstlisting}
% \\*
The version of MOOG currently is 2014 but can easily updated to newer versions.

\item
\begin{lstlisting}[language=Python]
save:         False
\end{lstlisting}
% \\*
If save = True, the output synthetic spectrum is saved in results.

\item
\begin{lstlisting}[language=Python]
teff:         False
logg:         False
feh:          False
vt:           False
vmac:         False
vsini:        False
\end{lstlisting}
% \\*
If any of these options is set, then they are fixed in the minimization process.

\item
\begin{lstlisting}[language=Python]
flag_vt:      False
flag_vmac:    False
\end{lstlisting}
% \\*
If any of these flags is on, the microturbulence or macroturbulence are set fixed to the values of empirical calibrations during the minimization process.

\item
\begin{lstlisting}[language=Python]
plot:         False
plot_res:     False
\end{lstlisting}
% \\*
These are plotting options of the synthetic/observed spectra (plot) and of their residuals in case of minimization (plot\_res).

\item
\begin{lstlisting}[language=Python]
damping:      1
\end{lstlisting}
% \\*
The damping option of MOOG to deal with the van der Waals broadening. The acceptable values are: 0, 1, 2.

\item
\begin{lstlisting}[language=Python]
step_wave:    0.01
step_flux:    3.0
\end{lstlisting}
% \\*
The wavelength step for the synthetic spectrum to be created and the flux limit to consider opacity distributions from the neighboring lines measured in \AA{}. These are the same options of MOOG.

\item
\begin{lstlisting}[language=Python]
minimize:     False
\end{lstlisting}
% \\*
The minimization option is on/off.

\item
\begin{lstlisting}[language=Python]
refine:       False
\end{lstlisting}
% \\*
The refine option performs corrections on the derived parameters after the minimization for a more detailed analysis. After a first run, a second minimization starts with optimized
microturbulence and macroturbulence values according to the parameters of the first run.

\item
\begin{lstlisting}[language=Python]
errors:       False
\end{lstlisting}
% \\*
The errors are calculated by shifting the free parameters by a certain value. If this option is of the errors correspond to the covariance matrix of the fit.

\item
\begin{lstlisting}[language=Python]
observations: False
\end{lstlisting}
% \\*
The name of the spectrum is given here. If the spectrum is at the spectra folder there is no need to give the full path.

\item
\begin{lstlisting}[language=Python]
inter_file:   intervals_hr10_15n.lst
\end{lstlisting}
% \\*
This is the file which contains the intervals of the synthesis. FASMA already has specific intervals which correspond to the given line list. This file should be inside the rawLinelist folder.

\item
\begin{lstlisting}[language=Python]
snr:          None
\end{lstlisting}
% \\*
The signal-to-noise (SNR) values of the observed spectra can be given by the user. They are used in the normalization process. Otherwise the SNR is calculated by the PyAstronomy function.

\item
\begin{lstlisting}[language=Python]
resolution:   None
\end{lstlisting}
% \\*
The resolution of the synthetic spectrum or the resolution of the observed spectrum. This value is necessary when minimization is on.

\item
\begin{lstlisting}[language=Python]
limb:         0.6
\end{lstlisting}
% \\*
The limb darkening coefficient for the vsini calculations. It is set fixed to 0.6 but can have different values depending on luminosity class.
\end{enumerate}


\section{Run FASMA}

There different ways to run FASMA. One way is to open the GUI with simple running:
\begin{lstlisting}[language=Python]
python FASMA.py
\end{lstlisting}
This will create the configuration file for you, but will only allow you to analyse one star.

To run the batch version (recommended) it can be done in two different ways. If you already have a
valid S``StarMe\_synth.cfg' file on your system, then simply run:
\begin{lstlisting}[language=Python]
python syntheDriver.py
\end{lstlisting}

If your configuration file is called something else (recommended for different projects), then run
\begin{lstlisting}[language=Python]
python synthDriver.py myConf.cfg
\end{lstlisting}


\end{document}
